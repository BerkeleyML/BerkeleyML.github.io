\documentclass[preview]{standalone}
\usepackage{header}
\begin{document}
\fontsize{12}{15}\selectfont

\Question{Getting Started}

\textbf{Read through this page carefully.} You may typeset your homework in latex or submit neatly handwritten/scanned solutions. Please start each question on a new page. Delivera$

\begin{enumerate}
  \item Submit a PDF of your writeup, \textbf{with an appendix for your code}, to assignment on Gradescope, ``<<title>> Write-Up". If there are graphs, include those graphs in the $
  \item If there is code, submit all code needed to reproduce your results, ``<<title>> Code".
  \item If there is a test set, submit your test set evaluation results, ``<<title>> Test Set".
\end{enumerate}

After you've submitted your homework, watch out for the self-grade form.

\begin{Parts}

\Part Who else did you work with on this homework? In case of course events, just describe the group. How did you work on this homework? Any comments about the homework?

\vspace{15pt}
\framebox(465, 75){}

\Part Please copy the following statement and sign next to it. We just want to make it \textit{extra} clear so that no one inadverdently cheats.

\textit{I certify that all solutions are entirely in my words and that I have not looked at another student's solutions. I have credited all external sources in this write up.}

\vspace{15pt}
\framebox(465, 75){}

\end{Parts}

\pagebreak


\newcommand{\choc}{\ensuremath{40}}
\newcommand{\chocp}{\ensuremath{39}}
\newcommand{\store}{\ensuremath{15}}
\Question{``Sample Complexity'' of coupon collecting}

One of the books of Roald Dahl has the following
problem. Willy Wonka has a chocolate factory where the produced
chocolates have $\choc$ different types of cards hidden under the
chocolate wrappers. To encourage people to consume these chocolates,
a game is announced: Whoever finds all the $\choc$ types of cards will
be allowed to enter Willy Wonka's factory and participate in a questionable
experiment. 

Charlie is a consumer of Willy's chocolates and he visits a particular
local store every day to buy his chocolates.
The store contains equal number of chocolates of each card type at any given time.
Every time a chocolate with a particular card hidden beneath the wrapper is bought, 
another chocolate containing an identical card is put in its place immediately.
Whenever Charlie buys a chocolate, he does that by picking up a chocolate uniformly at random from the store.

\begin{Parts}

\Part {\bf What is the expected number of days it takes Charlie to qualify for Willy Wonka's game if he buys one random chocolate every 
day from the local store?} Please explain your computations precisely.



\Part \emph{For all the following parts, the game has been changed.} 
Suppose there are $d$ types of different cards.
Instead of having to collect all $d$ card types, at the end of the game Willy 
Wonka will draw a card uniformly at random and if someone has that card in their collection,
they win a prize. Charlie still visits the local store everyday and buys a chocolate at random from it.
If Charlie wants his probability of winning to
be at least $1-\delta$, {\bf how many \emph{distinct} card types should he have in his
collection before Willy Wonka draws the card?}



\Part Suppose that Charlie visited the particular local store for $n$ days and bought
$1$ chocolate at random each day before Willy's draw of the random card.
{\bf What is the probability that Charlie wins a prize from Willy's draw?}



\Part 
Now assume $n = \alpha d$ (i.e. Charlie always buys exactly $\alpha$
times the number of total types of cards before Wonka's draw of the random
card).  {\bf What does Charlie's probability of winning converge to as
  $d$ gets large?  }



\Part Now, consider the following function estimation problem, which 
at first sight might seem unrelated. 
We want to learn a completely unstructured function $f$ on
a finite domain $\mathcal{D}$ of size $d$. We collect a 
training data of size $n$ from random samples,
i.e., we have the dataset $\{(U_i, f(U_i)), i=1, \ldots, n \}$ where
each $U_i$ is drawn uniformly at random from $\mathcal{D}$.
{\bf How big of a training set do we need to collect to ensure that 
with probability at least $1-\delta$, we will successfully
estimate the function at a point which is drawn uniformly at random from the domain $\mathcal{D}$?
Repeat the computations for the case when $d$ gets large.
}



\end{Parts}

\Question{The accuracy of learning decision boundaries}


This problem exercises your basic probability (e.g.~from 70) in the
context of understanding why lots of training data helps to improve the
accuracy of learning things.

For each $\theta \in (1/3,2/3)$, define $f_{\theta}: [0,1] \to \{0, 1\}$, such that
\begin{align}
f_\theta(x) =
\begin{cases}
1 \text{ if } x > \theta \notag\\
0 \text{ otherwise.} \label{eq:func}
\end{cases}
\end{align}
The function is plotted in Figure \ref{fig:func}.
\begin{figure}[h!]
    \begin{center}
    \includegraphics[width=0.5\textwidth]{src/problems/theory/f_fig.png}
    \caption{Plot of function $f_\theta(x)$ against $x$.} \label{fig:func}
    \end{center}
\end{figure}

We draw samples $X_1, X_2, \ldots, X_n$ uniformly at random and i.i.d. from the interval $[0,1]$.
Our goal is to learn an estimate for $\theta$ from $n$ random samples $(X_1, f_{\theta}(X_1)), (X_2, f_{\theta}(X_2)), \ldots, (X_n, f_{\theta}(X_n))$.


Let $T_{min} = \max(\{\frac{1}{3}\} \cup \{ X_i | f_\theta(X_i) = 0\})$. We know that the
true $\theta$ must be larger than $T_{min}$. 

Let $T_{max} = \min(\{\frac{2}{3}\} \cup \{ X_i | f_\theta(X_i) = 1\})$. We know that the
true $\theta$ must be smaller than $T_{max}$. 

The gap between $T_{min}$ and $T_{max}$ represents the uncertainty we
will have about the true $\theta$ given the training data that we have
received. 

\begin{Parts}

\Part \textbf{What is the probability that $T_{max} - \theta > \epsilon$ as a
function of $\epsilon$? And what is the probability that $\theta - T_{min} > \epsilon$ as a
function of $\epsilon$?}



\Part Suppose that you would like the estimator $\hat{\theta} = (T_{max} + T_{min}) / 2$ for $\theta$ that is $\epsilon$-close (defined as $|\hat{\theta} - \theta| < \epsilon$, where $\hat{\theta}$ is the estimation and $\theta$ is the true value) with probability at least
$1-\delta$. Both $\epsilon$ and $\delta$ are some small positive numbers. \textbf{Please bound or estimate how big of an $n$ do you need? } You do not need to find the optimal lowest sample complexity $n$, an approximation using results of question (a) is fine. 



\Part Let us say that instead of getting random samples $(X_i, f(X_i))$,
we were allowed to choose where to sample the function, but you had to
choose all the places you were going to sample in advance. \textbf{Propose a
method to estimate $\theta$. How many samples suffice to achieve an
estimate that is $\epsilon$-close as above?} ({\bf Hint:} You need not use a randomized
strategy.)  



\Part Suppose that you could pick where to sample the function
adaptively --- choosing where to sample the function in response to
what the answers were previously. \textbf{Propose a method to estimate
$\theta$. How many samples suffice to achieve an estimate that is
$\epsilon$-close as above?}



\Part In the three sampling approaches above: random, deterministic, and adaptive, \textbf{compare the scaling of $n$ with $\epsilon$ (and $\delta$ as well for the random case). }





\Part \textbf{Why do you think we asked this series of questions? What are the implications of those results in a machine learning application?}


















\end{Parts}
\Question{Eigenvalue and Eigenvector review}

A square matrix $\mat{A} \in \mathbb{R}^{d \times d}$ has a (right) eigenvalue $\lambda \in \mathbb{C}$
and (right) eigenvector $\vec{x} \in \mathbb{C}^{d} \setminus 0$ if $\mat{A}\vec{x} = \lambda \vec{x}$.
Left eigenvalues and eigenvectors are defined analogously --- $\vec{x}^T \mat{A} = \lambda \vec{x}^T$.
Since the definition is scale invariant
(if $\vec{x}$ is an eigenvector, then $t\vec{x}$ is an eigenvector for any $t \neq 0$),
we adopt the convention that each eigenvector has norm $1$.

\begin{Parts}

  \Part \textbf{Compute the right and left eigenvalues and
    eigenvectors} of the following matrices. You may use a computer
  for questions v) and vi).

i) $\mat{A} =
\begin{bmatrix}
2 & -4 \\
-1 & -1
\end{bmatrix}
$



ii) $\mat{B} =
\begin{bmatrix}
3 & 1 \\
1 & 3
\end{bmatrix}
$




iii) $\mat{A}^2$




iv) $\mat{B}^2$




v) $\mat{AB}$





vi) $\mat{BA}$




\Part \textbf{Compute the singular value decompositions} of the matrices
above. In addition, please compute the SVD of:
$\mat{C} =
\begin{bmatrix}
3 & 1 \\
1 & 3 \\
2 & -4 \\
-1 & -1
\end{bmatrix}
$. For $\mat{B}$ and $\mat{B}^2$ compute by hand. For the rest of the matrices you may
use a computer.




\Part \textbf{Show} from the definition of a right eigenvalue that the quantity
$\lambda$ is an eigenvalue with associated eigenvector $\vec{x}$ iff
for all $1 \leq i \leq d$, we have $$(\lambda - A_{ii}) x_i = \sum_{j
  \neq i} A_{ij} x_j.$$




\Part Now for an arbitrary eigenvalue $\lambda$ of $\mat{A}$ and its
associated eigenvector $\vec{x}$, choose index $i$ such that $|x_i|
\geq |x_j|$ for all $j \neq i$. For such an index $i$, \textbf{show} that
$$|\lambda - A_{ii}| \leq \sum_{j \neq i} |A_{ij}|.$$

You have just proved Gershgorin's circle theorem, which states that
all the eigenvalues of a $d \times d$ matrix lie within the union of
$d$ disks in the complex plane, where disk $i$ has center $A_{ii}$,
and radius $\sum_{j \neq i} |A_{ij}|$.




\end{Parts}



\Question{Fun with least squares}


In ordinary least squares we learn to predict a \emph{target} scalar
$y \in \real$ given a \emph{feature} vector
$\vec{x} \in \mathbb{R}^\vecdim$. Each element of $\vec{x}$ is called
a feature, which could correspond to a
scientific \emph{measurement}.
For example, the $i$-th element of
$\vec{x}$, denoted by $\el{x}{i}$, could correspond to the velocity of
a car at time $i$. $y$ could represent the final location (say just in
one direction) of the car.

For the purpose of predicting $y$ from $\vec{x}$ we are given $n$
samples $(\vec{x}_i, y_i)$ with $i = 1,\dots, n$ (where feature
vectors and target scalars are observed in pairs), which we also call
the training set. In this problem we want to predict the unobserved
target $y$ corresponding to a new $\vec{x}$ (not in the training set)
by some linear prediction $\hat{y} = \vec{x}^\top\widehat{\vecf{w}}$ where
the \emph{weight} $\widehat{\vecf{w}} \in \real^\vecdim$ minimizes the
least-squares training cost
\begin{equation*}
  \sum_{i=1}^n (\vec{x}_i^\top\vec{w} - y_i )^2 = \|\mat{X} \vec{w} - \vec{y} \|_2^2
\end{equation*}
where in the matrix $\mat{X} \in \real^{n\times d}$, the transposed
sample feature vectors $\vec{x}_i^\top$ constitute the $d$-dimensional row
vectors, and the $n$-dimensional vectors of training measurements
$\vec{x}^j = (\el{x_1}{j}, \dots, \el{x_n}{j})^\top$ for
$j = 1,\dots, \vecdim$ correspond to the column vectors (see
Figure~\ref{fig:Amat}).  and $\vec{y} = (y_1, \dots, y_n)^\top$.
\begin{figure}[h!]
  \centering
  \includegraphics[width=.7\textwidth]{src/problems/linear_regression/nestedleastsquaresFig.pdf}
  \caption{Dimensions, column and row vector notation for matrix $\mat{X}$}
  \label{fig:Amat}
\end{figure}


Let us actually build on the example mentioned above and view the
measurements $\el{x_i}{j}$ of each sample $\vec{x}_i$ as a sequence of
measurements, e.g. velocities of car $i$, over time
$j = 1,\dots, \vecdim$.

\begin{Parts}

  \Part Is this problem in a supervised or unsupervised learning setting?
      {\bf Please explain.} 

  
\Part Suppose that we want to learn (from our training set) to predict
the final location $y$ from only the first $\nummeas$ measurements.
Denoting the prediction of $y$ from the first $\nummeas$ measurements
by $\hat{y}^t$, we thus want to use $\el{x}{j}, j=1, \ldots, t$ to
predict $y$.  If we now learn how to obtain $\hat{y}^t$ for each
$t= 1,\dots,\vecdim$, we end up with a sequence of estimators
$\hat{y}^1,\dots, \hat{y}^\vecdim$ for each car.

  {\bf  Provide
    a method to obtain $\hat{y}^\nummeas$ for each $\nummeas$. }
  Note that we will obtain a different model for each $\nummeas$.
  



\Part Someone suggests that maybe the measurements themselves are
partially predictable from the previous measurements, which suggests
employing a two stage strategy to solve the original prediction
problem: First we predict the $\nummeas$-th measurement
$\el{x}{\nummeas}$ based on the previous measurements
$\el{x}{1},\dots, \el{x}{\nummeas-1}$. Then we look at the differences
(sometimes deemed the ``innovation'') between the actual $t$-th
measurement we obtained and our prediction for it, i.e.
$(\Delta \vec{ x})_{\nummeas} := (\vec{x})_{\nummeas} -
(\widehat{\vecf{x}})_{\nummeas}$ for $\nummeas > 1$ and assume
$(\Delta \vec{x})_1 = (\vec{x})_1$. Finally, we use
$(\Delta\vec{x})_1,\dots, (\Delta \vec{x})_{\nummeas}$ to obtain a
prediction $\tilde{y}^t$.

In order to learn the maps which allow us to (1) take
$\el{x}{1},\dots, \el{x}{\nummeas-1}$ to obtain
$(\Delta\vec{x})_1,\dots, (\Delta \vec{x})_{\nummeas}$ and (2) take
$(\Delta\vec{x})_1,\dots, (\Delta \vec{x})_{\nummeas}$ to predict
$\tilde{y}^t$, we again use our training set.  Specifically for each
$t$, in stage (1), we fit the vectors of training measurements
$\vec{x}^1, \dots, \vec{x}^{t-1}$ linearly to $\vec{x}^t$ using least
squares for each $t$. In stage (2), we use the innovation vectors
$(\Delta\vec{x}^1, \dots, \Delta\vec{x}^t)$ to predict $\vec{y}^t$
again using least squares.  Let's define the matrix
$\tilX^t := (\Delta\vec{x}^1, \dots, \Delta\vec{x}^t)$ and
$\tilX = \tilX^d$.


{\bf Show how} we can learn the best linear predictions
$\widehat{\vecf{x}}^\nummeas$ from
$\vec{x}^1,\dots, \vec{x}^{\nummeas-1}$. Then {\bf provide an
  expression} for $\tilde{\vec{y}}^t$ depending on the innovations
$\Delta\vec{x}^1,\dots, \Delta\vec{x}^{\nummeas}$.
  
  When presented with a new feature vector $\vec{x}$, are the sequence
  of final predictions of the one-stage training $\hat{y}^t$ in (b)
  and two-stage training $\tilde{y}^t$ in (c) the same?  {\bf Explain
    your reasoning.}
  





\Part  {\bf Which
  well-known procedure do
the steps to obtain $\tilX$ from $\mat{X}$ remind you of}?
({\bf HINT: } Think about how the column vectors in
$\tilX$ are geometrically related.)

{\bf Is there an efficient way to update the weight vector
  $\widehat{\vecf{w}}^{t}$ from $\widehat{\vecf{w}}^{t-1}$ when
  computing the sequence of predictions $\tilde{y}^t$?} Here,
$\widehat{\vecf{w}}^{t}$ are the weights for the second stage in (c).







\Part Now let's consider the more
general setting where we now want to predict a target vector $\vec{y} \in \real^k$
from a feature vector $\vec{x} \in \real^d$, thus having a training set
consisting of observations $(\vec{x}_i, \vec{y}_i)$ for $i=1,\dots,n$.

Instead of learning a weight vector $\vec{w} \in \real^\vecdim$, we
now want a linear estimate $\widehat{\vecf{y}} = \hat{\mat{W}} \vec{x}$
with a weight matrix $\hat{\mat{W}} \in \real^{k\times d}$
instead. From our samples, we obtain wide matrices $\mat{Y} \in \real^{k\times n}$ with
columns $\vec{y}_1,\dots,\vec{y}_n$ and $\mat{X} \in \real^{d\times n}$ with columns
$\vec{x}_1, \dots,\vec{x}_n$ (note that this is the transpose of
$\mat{X}$ in Figure~\ref{fig:Amat}!). In order to learn
$\hat{\mat{W}}$ we now want to minimize
$\|\mat{Y} - \mat{W}\mat{X}\|_F^2$ where $\|\cdot\|_F$ denotes the
Frobenius norm of matrices, i.e.
$\|\mat{L}\|_F^2 = \trace(\mat{L}^\top\mat{L})$.

{\bf Show how to find
$\hat{\mat{W}} = \arg \min_{\mat{W} \in \real^{k\times d}} \|\mat{Y} - \mat{W}\mat{X}\|_F^2$
using vector calculus as reviewed in Discussion 0 and 1.}




\Part In the setting of problem (e), {\bf argue why} the computation of the best
linear prediction $\widehat{\vecf{y}}$ of a target vector $\vec{y}$ using a feature
vector $\vec{x}$ can be solved by separately finding the best linear
prediction for each measurement $\el{y}{j}$ of the target vector $\vec{y}$.



\end{Parts}


\Question{System Identification by ordinary least squares regression}

Making autonomous vehicles involves machine learning for different purposes. One of which is
learning how cars actually behave based on their data.

\textbf{Make sure to submit the code you write in this problem to ``HW1 Code" on Gradescope.}

\begin{Parts}

\Part Consider the time sequence of scalars $x_t \in \mathbb{R}$ and $u_t \in \mathbb{R}$ in which
$x_{t+1} \approx A  x_{t} + B u_{t}$.  In control theory, $x_t$ usually represents the state, and
$u_t$ usually represents the control input.  \textbf{Find the numbers $A$ and $B$ so that
$\sum_t(x_{t+1} - A  x_{t} - B u_{t})^2$ is minimized.} The values of $x_t$ and $u_t$ are stored in
\texttt{a.mat}.




\Part Consider the time sequences of vectors
$\vec x_t \in \mathbb{R}^3$ and $\vec u_t \in \mathbb{R}^3$ in which
$\vec x_{t+1} \approx \mat A \vec x_{t} + \mat B \vec
u_{t}$. \textbf{Find the matrix $\mat A \in \mathbb{R}^{3\times 3}$
  and $\mat B \in \mathbb{R}^{3\times 3}$ so that the sum of the squared
  $\ell^2$-norms of the error,
  $\sum_t\lVert\vec x_{t+1} - \mat A \vec x_{t} - \mat B \vec
  u_{t}\lVert_2^2$, is minimized}.  The values of $\mat x_t$ and
$\mat u_t$ are stored in \texttt{b.mat}.




\Part Consider a \textit{car following model} that models how cars line up on a straight 1D highway
at a given time.  The acceleration of a car can be approximated by a linear function of the
positions and velocities of its own and the car in front of it.  Mathematically, we can formulate
this as $$ \ddot{x}_i \approx a x_{i} + b \dot{x}_{i} + c x_{i-1} + d \dot{x}_{i-1} + e,$$ where
$x_i$, $\dot x_i$, and $\ddot x_i$ are the position, velocity, and acceleration of the $i$th
car in the line.

\textbf{Find $a$, $b$, $c$, $d$, and $e$ that minimize
\[
\sum_i \left\lVert -\ddot{x}_i + a x_{i} + b \dot{x}_{i} + c x_{i-1} + d \dot{x}_{i-1} + e\right\lVert_2^2
\]
using data file \texttt{train.mat}}, which contains the status of 40\,000 cars at a given point
from the I-80 highway in California. The data were sampled from the Next Generation Simulation
(NGSIM) dataset so that the $i$ may not be continuous.  For your convenience, we give you the
profiles of each sampled car and the car that is in front of it.




\Part \textbf{Try to justify why your result in (c) is physically reasonable.}  Hint:  You can
reorganize your equation to be
\[
\ddot{x}_i = h (x_{i-1}- x_{i}) + f (\dot{x}_{i-1} - \dot{x}_{i}) - g (\dot{x}_{i}-L)+ w_i,
\]
and try to explain the physical meaning for each term, with $L$ being the speed limit.




\end{Parts}


\newcommand{\R}{\mathbb{R}}

\section{A Simple Classification Approach}

\textbf{Make sure to submit the code you write in this problem to ``HW1 Code'' on Gradescope.}

Classification is an important problem in applied machine
learning and is used in many different applications like image classification,
object detection, speech recognition, machine translation and others.

In \emph{classification}, we assign each datapoint a class from a finite set
(for example the image of a digit could be assigned the value $0, 1, \dots, 9$ of that digit).
This is different from \emph{regression}, where each datapoint is assigned a value from a
continuous domain like $\R$ (for example features of a house like location,
number of bedrooms, age of the house, etc. could be assigned the price of the house).

In this problem we consider the simplified setting of classification where we want to
classify data points from $\R^d$ into \emph{two} classes. For a linear classifier,
the space $\R^d$ is split into two parts by a hyperplane: All points on
one side of the hyperplane are classified as one class and all points on
the other side of the hyperplane are classified as the other class.

The goal of this problem is to
show that even a regression technique like linear regression can be used to solve a
classification problem. This can be
achieved by regressing the data points in the training set against $-1$ or $1$
depending on their class and then using the level set of 0 of the regression
function as the classification hyperplane (i.e. we use 0 as a threshold
on the output to decide between the classes).

Later in lecture we will learn why linear regression is not the optimal
approach for classification and we will study better approaches like logistic
regression, SVMs and neural networks.

\begin{Parts}

\Part The dataset used in this exercise is a subset of the MNIST dataset.
The MNIST dataset assigns
each image of a handwritten digit their value from 0 to 9 as a class.
For this problem we only keep digits that are assigned a 0 or 1, so we
simplify the problem to a two-class classification problem.

\textbf{Download and visualize the dataset (example code included). Include three
images that are labeled as 0 and three images that are labeled as 1
in your submission.}



\Part We now want to use linear regression for the problem, treating class
labels as real values $y = -1$ for class ``zero'' and $y = 1$ for class ``one''. In the dataset we
provide, the images have already been flattened into one dimensional vectors
(by concatenating all pixel values of the two dimensional image into a vector)
and stacked as rows into a feature matrix $\mat X$. We want to set up the regression
problem $\min_{\vec w}\norm{\mat X\vec w - \vec y}_2^2$ where the entry $y_i$ is the value of the
class ($-1$ or $1$) corresponding to the image in row $\vec{x}_i^\top$ of the feature matrix.
\textbf{Solve this regression problem for the training set and report the value of
$\norm{\mat X\vec w - \vec y}_2^2$ as well as the weights $\vec w$.} For this problem you
may only use pure Python and numpy (no machine learning libraries!).



\item Given a new flattened image $\vec x$, one natural rule to classify it is
the following one: It is a zero if $\vec x^\top \vec w \leq 0$ and a one if
$\vec x^\top \vec w > 0$. \textbf{Report what percentage of the digits in the training set
are correctly classified by this rule. Report what percentage of the digits in the test
set are correctly classified by this rule.}



\Part \textbf{Why is the performance typically evaluated on a separate test set
(instead of the training set) and why is the performance on the training and
test set similar in our case?} We will cover these questions in more detail
later in the class.


\Part Somebody suggests to use 0 (for class 0) and 1 (for class 1) as the
entries for the target vector $\vec b$. Try out how well this is doing
(make sure to adapt the classification rule, i.e. the threshold set for the outputs).
\textbf{Report what percentage of digits
are correctly classified using this approach on the training set and test set.}
How are the performances of the two approaches if you add a bias column
to the feature matrix $\mat X$? A bias column is a column of all ones, i.e. the new
feature matrix $\mat{X'}$ is
$$\mat{X'} = \begin{bmatrix}\vec{x}_0^\top & 1\\\vec{x}_1^\top & 1\\ \vdots & \vdots \\ \vec{x}_n^\top & 1 \end{bmatrix}$$
\textbf{Report what percentage of digits
are correctly classified using regression targets $0/1$ and $-1/1$ with bias on the training set and test set.}
\textbf{Try to explain the results!}





\end{Parts}

\Question{Your Own Question}

{\bf Write your own question, and provide a thorough solution.}

Writing your own problems is a very important way to really learn
the material. The famous ``Bloom's Taxonomy'' that lists the levels of
learning is: Remember, Understand, Apply, Analyze, Evaluate, and
Create. Using what you know to create is the top-level. We rarely ask
you any HW questions about the lowest level of straight-up
remembering, expecting you to be able to do that yourself. (e.g. make
yourself flashcards) But we don't want the same to be true about the
highest level.

As a practical matter, having some practice at trying to create
problems helps you study for exams much better than simply counting on
solving existing practice problems. This is because thinking about how
to create an interesting problem forces you to really look at the
material from the perspective of those who are going to create the
exams. 

Besides, this is fun. If you want to make a boring problem, go
ahead. That is your prerogative. But it is more fun to really engage
with the material, discover something interesting, and then come up
with a problem that walks others down a journey that lets them share
your discovery. You don't have to achieve this every week. But unless
you try every week, it probably won't happen ever. 

\end{document}
